\documentclass[man]{apa6}
\usepackage{lmodern}
\usepackage{amssymb,amsmath}
\usepackage{ifxetex,ifluatex}
\usepackage{fixltx2e} % provides \textsubscript
\ifnum 0\ifxetex 1\fi\ifluatex 1\fi=0 % if pdftex
  \usepackage[T1]{fontenc}
  \usepackage[utf8]{inputenc}
\else % if luatex or xelatex
  \ifxetex
    \usepackage{mathspec}
  \else
    \usepackage{fontspec}
  \fi
  \defaultfontfeatures{Ligatures=TeX,Scale=MatchLowercase}
\fi
% use upquote if available, for straight quotes in verbatim environments
\IfFileExists{upquote.sty}{\usepackage{upquote}}{}
% use microtype if available
\IfFileExists{microtype.sty}{%
\usepackage{microtype}
\UseMicrotypeSet[protrusion]{basicmath} % disable protrusion for tt fonts
}{}
\usepackage{hyperref}
\hypersetup{unicode=true,
            pdftitle={Påvirker tyggis konsentrasjonen i klasserommet?},
            pdfauthor={Henrik Sørlie, Astrid Lenvik, Frida Kathrine Sofie Mathisen, Helga Bjørnøy Urke, Torbjørn Torsheim, Thomas Potrebny , Marit Larsen, sylwia.kolasa, \& Thomas Hol Fosse},
            pdfkeywords={tyggis, konsentrasjon, klasserom, skole},
            pdfborder={0 0 0},
            breaklinks=true}
\urlstyle{same}  % don't use monospace font for urls
\usepackage{graphicx,grffile}
\makeatletter
\def\maxwidth{\ifdim\Gin@nat@width>\linewidth\linewidth\else\Gin@nat@width\fi}
\def\maxheight{\ifdim\Gin@nat@height>\textheight\textheight\else\Gin@nat@height\fi}
\makeatother
% Scale images if necessary, so that they will not overflow the page
% margins by default, and it is still possible to overwrite the defaults
% using explicit options in \includegraphics[width, height, ...]{}
\setkeys{Gin}{width=\maxwidth,height=\maxheight,keepaspectratio}
\IfFileExists{parskip.sty}{%
\usepackage{parskip}
}{% else
\setlength{\parindent}{0pt}
\setlength{\parskip}{6pt plus 2pt minus 1pt}
}
\setlength{\emergencystretch}{3em}  % prevent overfull lines
\providecommand{\tightlist}{%
  \setlength{\itemsep}{0pt}\setlength{\parskip}{0pt}}
\setcounter{secnumdepth}{0}
% Redefines (sub)paragraphs to behave more like sections
\ifx\paragraph\undefined\else
\let\oldparagraph\paragraph
\renewcommand{\paragraph}[1]{\oldparagraph{#1}\mbox{}}
\fi
\ifx\subparagraph\undefined\else
\let\oldsubparagraph\subparagraph
\renewcommand{\subparagraph}[1]{\oldsubparagraph{#1}\mbox{}}
\fi

%%% Use protect on footnotes to avoid problems with footnotes in titles
\let\rmarkdownfootnote\footnote%
\def\footnote{\protect\rmarkdownfootnote}


  \title{Påvirker tyggis konsentrasjonen i klasserommet?}
    \author{Henrik Sørlie\textsuperscript{1,2}, Astrid Lenvik\textsuperscript{2}, Frida Kathrine Sofie Mathisen\textsuperscript{2}, Helga Bjørnøy Urke\textsuperscript{2}, Torbjørn Torsheim\textsuperscript{2}, Thomas Potrebny \textsuperscript{3}, Marit Larsen\textsuperscript{3}, sylwia.kolasa\textsuperscript{3}, \& Thomas Hol Fosse\textsuperscript{1}}
    \date{}
  
\shorttitle{TYGGIS OG KONSENTRASJON}
\affiliation{
\vspace{0.5cm}
\textsuperscript{1} Forsvarets høgskole\\\textsuperscript{2} Universitetet i Bergen\\\textsuperscript{3} Hoegskulen på Vestlandet\\\textsuperscript{4} Norce}
\keywords{tyggis, konsentrasjon, klasserom, skole\newline\indent Word count: X}
\usepackage{csquotes}
\usepackage{upgreek}
\captionsetup{font=singlespacing,justification=justified}

\usepackage{longtable}
\usepackage{lscape}
\usepackage{multirow}
\usepackage{tabularx}
\usepackage[flushleft]{threeparttable}
\usepackage{threeparttablex}

\newenvironment{lltable}{\begin{landscape}\begin{center}\begin{ThreePartTable}}{\end{ThreePartTable}\end{center}\end{landscape}}

\makeatletter
\newcommand\LastLTentrywidth{1em}
\newlength\longtablewidth
\setlength{\longtablewidth}{1in}
\newcommand{\getlongtablewidth}{\begingroup \ifcsname LT@\roman{LT@tables}\endcsname \global\longtablewidth=0pt \renewcommand{\LT@entry}[2]{\global\advance\longtablewidth by ##2\relax\gdef\LastLTentrywidth{##2}}\@nameuse{LT@\roman{LT@tables}} \fi \endgroup}


\DeclareDelayedFloatFlavor{ThreePartTable}{table}
\DeclareDelayedFloatFlavor{lltable}{table}
\DeclareDelayedFloatFlavor*{longtable}{table}
\makeatletter
\renewcommand{\efloat@iwrite}[1]{\immediate\expandafter\protected@write\csname efloat@post#1\endcsname{}}
\makeatother
\usepackage{lineno}

\linenumbers

\authornote{Forfatterne fraskiver seg etterhvert ansvar for at resultatene ikke stemmer med virkeligheten i resten av verden.

Correspondence concerning this article should be addressed to Henrik Sørlie, Akershus festning, Oslo. E-mail: \href{mailto:henrik.sorlie@uib.no}{\nolinkurl{henrik.sorlie@uib.no}}}

\abstract{
Her står et velskrevet sammendrag som gir en god oversikt over studiens formål, hypoteser og resultater.


}

\begin{document}
\maketitle

\hypertarget{introduksjon}{%
\section{Introduksjon}\label{introduksjon}}

God skolegjennomføring er viktig for barnas senere muligheter i arbeidslivet.
Skolehverdagen kan være hektisk der barna skal utføre oppgaver i et miljø fylt av mulige distraksjoner(Tänzer, von Fintel, \& Eikermann, 2009).
Dette kan hindre læring, og studier viser at barnas evne til å konsentrere seg er en viktig faktor for økt mestring og læring.\\
Ro i klasserommet, disipin er faktorer som man vet fremmer konsentrasjon og økt læring.
En faktor som er lite undersøkt, men som en vet fra andre områder er fremmende for konsentrasjon, er tyggis.
Tidligere forskning har vist at tyggis kan være et enkelt og effektivt middel for å fremme konsentrasjon, redusere stress, ført til at man kan bli mer oppvakt og øke kognisjon (Allen \& Smith, 2011)(Tänzer et al., 2009).Tyggisvirker dermimot ikke å være like effektivt blant barn med tillegsutfordringer, som feks ADHD (Tucha et al., 2010)

\hypertarget{konsentrasjon-og-tyggis}{%
\subsection{Konsentrasjon og tyggis}\label{konsentrasjon-og-tyggis}}

Studier viser det å tygge et stykke sukkerfritt tyggegummi forbedrer hukommelse og oppmerksomhetsfunksjoner hos friske barn og voksne i ulike situasjoner (Baker, Bezance, Zellaby, \& Aggleton, 2004; Scholey et al., 2009; Tänzer et al., 2009; Wilkinson, Scholey, \& Wesnes, 2002). Dette er ikke tidligere undersøkt mindre barn og det etterlyses mer forskning på dette.
På denne bakgrunn utledes følgende hypoteser:

Hypotese 1. Tyggistygging i skoletimene på barneskolen øker konsentrasjonen

Hypotese 2: Tyggistygging i skoletimene fører til økt ro i klasserommet.

\hypertarget{method}{%
\section{Method}\label{method}}

\hypertarget{participants}{%
\subsection{Participants}\label{participants}}

\hypertarget{material}{%
\subsection{Material}\label{material}}

\hypertarget{procedure}{%
\subsection{Procedure}\label{procedure}}

\hypertarget{data-analysis}{%
\subsection{Data analysis}\label{data-analysis}}

We used R (Version 3.6.2; R Core Team, 2019) and the R-packages \emph{dplyr} (Version 0.8.3; Wickham et al., 2019), \emph{forcats} (Version 0.4.0; Wickham, 2019a), \emph{ggplot2} (Version 3.2.1; Wickham, 2016), \emph{papaja} (Version 0.1.0.9842; Aust \& Barth, 2020), \emph{purrr} (Version 0.3.3; Henry \& Wickham, 2019), \emph{readr} (Version 1.3.1; Wickham, Hester, \& Francois, 2018), \emph{stringr} (Version 1.4.0; Wickham, 2019b), \emph{tibble} (Version 2.1.3; Müller \& Wickham, 2019), \emph{tidyr} (Version 1.0.0; Wickham \& Henry, 2020), and \emph{tidyverse} (Version 1.2.1; Wickham, 2017) for all our analyses.

\hypertarget{results}{%
\section{Results}\label{results}}

\hypertarget{discussion}{%
\section{Discussion}\label{discussion}}

\newpage

\hypertarget{references}{%
\section{References}\label{references}}

\begingroup
\setlength{\parindent}{-0.5in}
\setlength{\leftskip}{0.5in}

\hypertarget{refs}{}
\leavevmode\hypertarget{ref-allenReviewEvidenceThat2011}{}%
Allen, A. P., \& Smith, A. P. (2011). A Review of the Evidence that Chewing Gum Affects Stress, Alertness and Cognition, \emph{9}, 18.

\leavevmode\hypertarget{ref-R-papaja}{}%
Aust, F., \& Barth, M. (2020). \emph{papaja: Create APA manuscripts with R Markdown}. Retrieved from \url{https://github.com/crsh/papaja}

\leavevmode\hypertarget{ref-bakerChewingGumCan2004}{}%
Baker, J. R., Bezance, J. B., Zellaby, E., \& Aggleton, J. P. (2004). Chewing gum can produce context-dependent effects upon memory. \emph{Appetite}, \emph{43}(2), 207--210. doi:\href{https://doi.org/10.1016/j.appet.2004.06.004}{10.1016/j.appet.2004.06.004}

\leavevmode\hypertarget{ref-R-purrr}{}%
Henry, L., \& Wickham, H. (2019). \emph{Purrr: Functional programming tools}. Retrieved from \url{https://CRAN.R-project.org/package=purrr}

\leavevmode\hypertarget{ref-R-tibble}{}%
Müller, K., \& Wickham, H. (2019). \emph{Tibble: Simple data frames}. Retrieved from \url{https://CRAN.R-project.org/package=tibble}

\leavevmode\hypertarget{ref-R-base}{}%
R Core Team. (2019). \emph{R: A language and environment for statistical computing}. Vienna, Austria: R Foundation for Statistical Computing. Retrieved from \url{https://www.R-project.org/}

\leavevmode\hypertarget{ref-scholeyChewingGumAlleviates2009}{}%
Scholey, A., Haskell, C., Robertson, B., Kennedy, D., Milne, A., \& Wetherell, M. (2009). Chewing gum alleviates negative mood and reduces cortisol during acute laboratory psychological stress. \emph{Physiology \& Behavior}, \emph{97}(3), 304--312. doi:\href{https://doi.org/10.1016/j.physbeh.2009.02.028}{10.1016/j.physbeh.2009.02.028}

\leavevmode\hypertarget{ref-tanzerChewingGumConcentration2009}{}%
Tänzer, U., von Fintel, A., \& Eikermann, T. (2009). Chewing Gum and Concentration Performance. \emph{Psychological Reports}, \emph{105}(2), 372--374. doi:\href{https://doi.org/10.2466/PR0.105.2.372-374}{10.2466/PR0.105.2.372-374}

\leavevmode\hypertarget{ref-tuchaDetrimentalEffectsGum2010}{}%
Tucha, L., Simpson, W., Evans, L., Birrel, L., Sontag, T. A., Lange, K. W., \& Tucha, O. (2010). Detrimental effects of gum chewing on vigilance in children with attention deficit hyperactivity disorder. \emph{Appetite}, \emph{55}(3), 679--684. doi:\href{https://doi.org/10.1016/j.appet.2010.10.001}{10.1016/j.appet.2010.10.001}

\leavevmode\hypertarget{ref-R-ggplot2}{}%
Wickham, H. (2016). \emph{Ggplot2: Elegant graphics for data analysis}. Springer-Verlag New York. Retrieved from \url{https://ggplot2.tidyverse.org}

\leavevmode\hypertarget{ref-R-tidyverse}{}%
Wickham, H. (2017). \emph{Tidyverse: Easily install and load the 'tidyverse'}. Retrieved from \url{https://CRAN.R-project.org/package=tidyverse}

\leavevmode\hypertarget{ref-R-forcats}{}%
Wickham, H. (2019a). \emph{Forcats: Tools for working with categorical variables (factors)}. Retrieved from \url{https://CRAN.R-project.org/package=forcats}

\leavevmode\hypertarget{ref-R-stringr}{}%
Wickham, H. (2019b). \emph{Stringr: Simple, consistent wrappers for common string operations}. Retrieved from \url{https://CRAN.R-project.org/package=stringr}

\leavevmode\hypertarget{ref-R-dplyr}{}%
Wickham, H., François, R., Henry, L., \& Müller, K. (2019). \emph{Dplyr: A grammar of data manipulation}. Retrieved from \url{https://CRAN.R-project.org/package=dplyr}

\leavevmode\hypertarget{ref-R-tidyr}{}%
Wickham, H., \& Henry, L. (2020). \emph{Tidyr: Tidy messy data}. Retrieved from \url{https://CRAN.R-project.org/package=tidyr}

\leavevmode\hypertarget{ref-R-readr}{}%
Wickham, H., Hester, J., \& Francois, R. (2018). \emph{Readr: Read rectangular text data}. Retrieved from \url{https://CRAN.R-project.org/package=readr}

\leavevmode\hypertarget{ref-wilkinsonChewingGumSelectively2002}{}%
Wilkinson, L., Scholey, A., \& Wesnes, K. (2002). Chewing gum selectively improves aspects of memory in healthy volunteers. \emph{Appetite}, \emph{38}(3), 235--236. doi:\href{https://doi.org/10.1006/appe.2002.0473}{10.1006/appe.2002.0473}

\endgroup


\end{document}
